\documentclass{article}
\usepackage[utf8]{inputenc}
\usepackage{xspace}

\newcommand{\editor}{\mbox{\textsf{SNE}}}

\title{\editor  ~-- The Neurion editor}
\author{Hans Hüttel \and Michael Koldsgaard \and Thomas Munk}
\date{August 2021}

\begin{document}

\maketitle

\section{Introduction}

This document outlines the design decisions underlying \editor.

\section{Handles and environments}

Every value in the knowledge base is bound by a name, which we call it
its \emph{handle}. Handles are unique; the collection of values known
is therefore an \emph{environment} in the usual sense of program
semantics.

The environment is represented as a database.

We distinguish between internal and external handles. Internal handles
are used by the database and are necessarily unique. External handles
are used by the programmer and may not be unique.

There are various approaches to representing internal
handles. deBruijn indexes are examples of internal handles. Another
possibility is to use sequences of atomic handles that denote nested
scopes such that e.g. $a_3.a_7.a_2$ denotes the value with atomic
handle $a_2$ found within the scope with atomic handle $a_7$, that is
found within the scope with atomic handle $a_3$.

\section{Edit actions}

We distinguish between edit actions that modify values, actions
that modify the environment and actions that modify annotations.

\subsection{Editing values}

The editor supports the following atomic edit actions for values.

\begin{itemize}
\item Introduce an AST
\item Introduce AST nodes
\item Delete AST nodes
\end{itemize}

These correspond to the atomic actions found in the editor calculus of 
\cite{DBLP:conf/pepm/GodiksenHHLO21}. Composite actions can be defined
using the control structures in the editor calculus. An example is
that deletion can be defined as substituing $a$ by the hole $[\;]$.

\subsection{Editing annotations}

Values can be annotated. Most notably, values can be annotated with
types. Type annotations can be derived using type inference, but we
can also explicitly type a value by declaring the type of its handle.

\section{How edit actions modify the data}


Håndtag skal have et tidsstempel – eller "den ældste committede"/"nyeste committede"

En save-knap.

\section{The implementation}

(men kan det understøttes pænt i Elm?)

** Det gode ved den nuværende implementation: Få primitiver
** Det mindre gode: Mange specialiseringer (pga. manglende ad hoc-polymorfi)


\section{Future work}

Re-implement everything in NPL.

\bibliographystyle{plain}
\bibliography{editor}

\end{document}

%%% Local Variables:
%%% mode: latex
%%% TeX-master: t
%%% End:
