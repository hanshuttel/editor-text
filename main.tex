\documentclass{article}
\usepackage[utf8]{inputenc}

\title{The Neurion editor}
\author{Hans Hüttel \and Michael Koldsgaard}
\date{July 2021}

\begin{document}

\maketitle

\section{Introduction}

The goal of this 

\section{About handles}

The association between handles and values (the set of bindings) is an environment in the usual sense of program semantics.

Some actions modify values. Others modify handles and the association between handles and values.

Handles can be internal or external. deBruijn indexes are examples of internal handles.

\section{Edit actions}

The editor supports

\begin{itemize}
    \item Operationer på AST'er
\begin{itemize}
    \item Indsætte AST, tilføje/slette søskendeknuder
  
** Flere niveauer af data

  *** Samlingen af data som et AST
  *** Et enkelt datum som et AST

\end{itemize}
 
\item Sammensatte handlinger kan defineres ved makroer (vis hvordan -
  f.eks. kan sletning udtrykkes som at substituere med et hul)

    
\end{itemize}


Alt dette kræver et sprog for AST'er (det er NPLs syntaks med huller
tilføjet)


* Typeannoteringer i editoren (annotere knuder med deres typer -
skal det bare være "by need", så man laver typeinferens, når man skal
annotere? Det er nok bedst sådan, men bør overvejes)

\section{How edit actions modify the data}

Alle data har et håndtag. Håndtag skal være entydige (Hvordan skal scopes repræsenteres? Lambda-lifting? -- relevant overvejelse for krydshenvisninger i forbindelse med modularisering)

Håndtag skal have et tidsstempel – eller "den ældste committede"/"nyeste committede"

En save-knap.

\section{The implementation}

(men kan det understøttes pænt i Elm?)

** Det gode ved den nuværende implementation: Få primitiver
** Det mindre gode: Mange specialiseringer (pga. manglende ad hoc-polymorfi)


\section{Future work}

En skønne dag: Genimplementere editoren i NPL


\end{document}
